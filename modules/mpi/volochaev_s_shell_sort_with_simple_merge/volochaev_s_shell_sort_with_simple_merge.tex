\documentclass[a4paper,12pt]{article}
\usepackage[utf8]{inputenc}
\usepackage[russian]{babel}
\usepackage{amsmath, amssymb}
\usepackage{graphicx}
\usepackage{listings}
\usepackage{color}
\usepackage{hyperref}
\usepackage{geometry}
\usepackage{caption}
\usepackage{float}

\geometry{left=3cm, right=2cm, top=2cm, bottom=2cm}

\definecolor{codegreen}{rgb}{0,0.6,0}
\definecolor{codegray}{rgb}{0.5,0.5,0.5}
\definecolor{codepurple}{rgb}{0.58,0,0.82}
\definecolor{backcolour}{rgb}{0.95,0.95,0.92}

\lstdefinestyle{mystyle}{
    backgroundcolor=\color{backcolour},   
    commentstyle=\color{codegreen},
    keywordstyle=\color{blue},
    numberstyle=\tiny\color{codegray},
    stringstyle=\color{codepurple},
    basicstyle=\ttfamily\footnotesize,
    breakatwhitespace=false,         
    breaklines=true,                 
    captionpos=b,                    
    keepspaces=true,                 
    numbers=left,                    
    numbersep=5pt,                  
    showspaces=false,                
    showstringspaces=false,
    showtabs=false,                  
    tabsize=2
}
\begin{document}
\begin{titlepage}
    \begin{center}
        \large
        \textbf{ННГУ им. Лобачевского / ИИТММ / ПМИ}\\[0.5cm]

        \vspace{4cm}
        \textbf{\Large Отчёт по выполнению задания}\\
        \textbf{\large <<Сортироовка Шелла с простым слиянием с использованием MPI>>}\\[3cm]

        \vspace{3cm}
        \textbf{Выполнил:}\\
        студент группы 3822Б1ПМоп3 \\
        \textit{Волочаев Серафим Витальевич}\\[1cm]

        \textbf{Преподаватель:}\\
        \textit{Сысоев Александр Владимирович, доцент}\\[2cm]

        \vfill
        \textbf{Нижний Новгород, 2024 г.}
    \end{center}
\end{titlepage}
\date{\today}

\begin{document}

\maketitle

\tableofcontents
\newpage

\section{Введение}
Сортировка является одной из фундаментальных операций в информатике и играет ключевую роль во множестве приложений, начиная от обработки данных до алгоритмов машинного обучения. В условиях роста объёмов данных и требований к быстродействию, эффективные алгоритмы сортировки, способные использовать параллельные вычислительные ресурсы, становятся особенно актуальными.

\section{Постановка задачи}
Целью данной работы является разработка и анализ параллельного алгоритма сортировки на основе метода Shell Sort с использованием технологии MPI (Message Passing Interface). Задача состоит в реализации эффективного алгоритма, способного сортировать большие массивы данных, используя преимущества параллельных вычислений для сокращения времени выполнения и обеспечения масштабируемости.

\section{Описание алгоритма}
Алгоритм Shell Sort является обобщением сортировки вставками, который использует последовательность уменьшения шага (\textit{gap}) для сортировки элементов, удалённых друг от друга на заданное расстояние. Начальное значение шага выбирается как половина длины массива, и постепенно уменьшается до единицы, при этом на каждом этапе выполняется сортировка вставками с текущим шагом.

В данной реализации после выполнения Shell Sort для каждой части массива, локально отсортированные подмассивы сливаются в один отсортированный массив с помощью стандартного алгоритма слияния, обеспечивая тем самым глобальную сортировку массива.

\section{Описание схемы распараллеливания}
Параллельная версия алгоритма использует технологию MPI для распределения работы между несколькими процессами. Схема распараллеливания включает следующие этапы:

\begin{enumerate}
    \item \textbf{Инициализация:} Процесс-ведущий (ранг 0) инициализирует массив данных и распределяет его части между всеми доступными процессами.
    \item \textbf{Локальная сортировка:} Каждый процесс выполняет сортировку своей части массива с использованием последовательного алгоритма Shell Sort.
    \item \textbf{Слияние:} Отсортированные подмассивы собираются обратно процессом-ведущим, который выполняет последовательное слияние всех полученных частей в один полностью отсортированный массив.
    \item \textbf{Финализация:} Отсортированный массив передаётся обратно всем процессам или сохраняется, в зависимости от требований приложения.
\end{enumerate}

\section{Описание программной реализации}
Реализация параллельного алгоритма сортировки выполнена на языке C++ с использованием библиотеки Boost.MPI, которая обеспечивает удобный интерфейс для работы с MPI. Основные компоненты реализации включают:

\begin{itemize}
    \item \textbf{Функция \texttt{sort}:} Реализует последовательный алгоритм Shell Sort для локальной сортировки подмассивов.
    \item \textbf{Функция \texttt{merge}:} Выполняет слияние двух отсортированных векторов в один.
    \item \textbf{Класс \texttt{Lab3\_16\_mpi}:} Оборачивает основные этапы алгоритма, включая предобработку, валидацию, выполнение и постобработку. Использует MPI для распределения данных и сбора результатов.
\end{itemize}

\subsection{Основные функции}

\begin{itemize}
    \item \texttt{pre\_processing():} Инициализирует данные для сортировки, распределяя массив между процессами.
    \item \texttt{run():} Выполняет локальную сортировку и собирает результаты с помощью слияния.
    \item \texttt{post\_processing():} Сохраняет отсортированные данные в выходной массив.
\end{itemize}
    
\section{Результаты экспериментов}
Для оценки эффективности разработанного алгоритма были проведены эксперименты по измерению времени выполнения и проверке корректности сортировки. Эксперименты проводились на различных размерах массивов и количестве процессов.

\subsection{Время выполнения}
\begin{table}[H]
    \centering
    \caption{Время выполнения алгоритма Shell Sort слиянием при различных размерах массива и количестве процессов}
    \begin{tabular}{|c|c|c|c|c|}
        \hline
        Размер массива & 1 процесс & 2 процесса & 4 процесса & 8 процессов \\
        \hline
        $10^5$ & 0.0145467999 с& 0.0123804000 с& 0.0126331000 с& 0.0184341000 с\\
        \hline
        $10^6$ & 0.1870559000 с& 0.1226348000 с& 0.1102627000 с& 0.1612291000 с\\
        \hline
        $10^7$ & 2.4188886001 с& 1.6767619000 с& 1.4307465000 с& 1.4413447999 с\\
        \hline
    \end{tabular}
\end{table}

\subsection{Качество работы алгоритма}
Алгоритм продемонстрировал корректность сортировки на всех тестируемых размерах массивов. Для подтверждения корректности был использован стандартный метод проверки отсортированного массива на неубывающий порядок.

\subsection{Описание подтверждения корректности}
После выполнения сортировки, каждый процесс проверял свой локальный подмассив на соответствие сортировочному порядку. Дополнительно, процесс-ведущий выполнял окончательное слияние и проверял весь массив на корректность глобальной сортировки.

\section{Выводы из результатов}
Полученные результаты подтверждают эффективность параллельной реализации алгоритма Shell Sort слиянием на основе MPI. Увеличение числа процессов приводит к снижению времени выполнения, демонстрируя хорошую масштабируемость алгоритма. Корректность сортировки была подтверждена на всех тестируемых размерах данных.

\section{Заключение}
В данной работе был разработан и реализован параллельный алгоритм сортировки на основе Shell Sort с использованием MPI. Экспериментальные результаты показали, что алгоритм эффективно использует параллельные вычислительные ресурсы, обеспечивая значительное сокращение времени выполнения при увеличении числа процессов. Дальнейшие исследования могут быть направлены на оптимизацию алгоритма слияния и исследование других стратегий распределения данных для повышения производительности.

\section{Литература}
\begin{enumerate}
    \item Гергель В.П. \textit{Введение Введение в методы параллельного параллельного программирования},  \url {http://www.hpcc.unn.ru/mskurs/RUS/PPT/ppr04_2.pdf}
    \item Boost MPI Library Documentation, \url{https://www.boost.org/doc/libs/release/libs/mpi/doc/index.html}
    \item  Учебное пособие ННГУ по MPI \url{http://www.hpcc.unn.ru/mskurs/RUS/DOC/ppr04.pdf}
    \item Видео практик, \url{https://disk.yandex.ru/d/LxDwnOlUo3Eqfg}
\end{enumerate}

\newpage
\appendix
\section{Исходный код}
\begin{lstlisting}[language=C++, caption={Реализация параллельного алгоритма сортировки на основе Shell Sort слиянием}, label=lst:code]
#include "mpi/volochaev_s_shell_sort_with_simple_merge_16/include/ops_mpi.hpp"

#include <algorithm>
#include <boost/mpi.hpp>
#include <boost/serialization/vector.hpp>
#include <functional>
#include <thread>
#include <vector>

namespace volochaev_s_shell_sort_with_simple_merge_16_mpi {
void sort(std::vector<int>& v);
std::vector<int> merge(const std::vector<int>& left, const std::vector<int>& right);
}  // namespace volochaev_s_shell_sort_with_simple_merge_16_mpi

bool volochaev_s_shell_sort_with_simple_merge_16_mpi::Lab3_16_seq::pre_processing() {
  internal_order_test();

  int* x = reinterpret_cast<int*>(taskData->inputs[0]);
  size_ = static_cast<int>(taskData->inputs_count[0]);
  mas.resize(size_);

  std::copy(x, x + size_, mas.begin());

  return true;
}

bool volochaev_s_shell_sort_with_simple_merge_16_mpi::Lab3_16_seq::validation() {
  internal_order_test();

  return static_cast<int>(taskData->inputs_count[0]) > 0;
}

bool volochaev_s_shell_sort_with_simple_merge_16_mpi::Lab3_16_seq::run() {
  internal_order_test();

  int n = size_;
  int gap = n / 2;

  while (gap > 0) {
    for (int i = gap; i < n; ++i) {
      int temp = mas[i];
      int j = i;
      while (j >= gap && mas[j - gap] > temp) {
        mas[j] = mas[j - gap];
        j -= gap;
      }
      mas[j] = temp;
    }
    gap /= 2;
  }

  return true;
}

bool volochaev_s_shell_sort_with_simple_merge_16_mpi::Lab3_16_seq::post_processing() {
  internal_order_test();

  int* ans = reinterpret_cast<int*>(taskData->outputs[0]);
  std::copy(mas.begin(), mas.end(), ans);
  return true;
}

bool volochaev_s_shell_sort_with_simple_merge_16_mpi::Lab3_16_mpi::pre_processing() {
  internal_order_test();

  if (world.rank() == 0) {
    int* x = reinterpret_cast<int*>(taskData->inputs[0]);
    size_ = taskData->inputs_count[0];

    mas.resize(size_);

    std::copy(x, x + size_, mas.begin());
  }

  return true;
}

bool volochaev_s_shell_sort_with_simple_merge_16_mpi::Lab3_16_mpi::validation() {
  internal_order_test();
  if (world.rank() == 0) {
    return static_cast<int>(taskData->inputs_count[0]) > 0;
  }

  return true;
}

void volochaev_s_shell_sort_with_simple_merge_16_mpi::sort(std::vector<int>& v) {
  int n = v.size();
  int gap = n / 2;

  while (gap > 0) {
    for (int i = gap; i < n; ++i) {
      int temp = v[i];
      int j = i;
      while (j >= gap && v[j - gap] > temp) {
        v[j] = v[j - gap];
        j -= gap;
      }
      v[j] = temp;
    }
    gap /= 2;
  }
}

std::vector<int> volochaev_s_shell_sort_with_simple_merge_16_mpi::merge(const std::vector<int>& left,
                                                                        const std::vector<int>& right) {
  std::vector<int> res;
  res.reserve(left.size() + right.size());
  std::merge(left.begin(), left.end(), right.begin(), right.end(), std::back_inserter(res));

  return res;
}

bool volochaev_s_shell_sort_with_simple_merge_16_mpi::Lab3_16_mpi::run() {
  internal_order_test();

  broadcast(world, size_, 0);
  std::vector<int> sizes;

  if (world.rank() == 0) {
    int delta = size_ / world.size();
    int ost = size_ % world.size();

    sizes.resize(world.size(), delta);
    for (int i = 0; i < ost; ++i) {
      ++sizes[i];
    }
  }

  broadcast(world, sizes, 0);

  local_input.resize(sizes[world.rank()]);
  scatterv(world, mas.data(), sizes, local_input.data(), 0);

  sort(local_input);

  if (world.rank() == 0) {
    mas = local_input;
    std::vector<int> data;

    for (int i = 1; i < world.size(); ++i) {
      world.recv(i, 0, data);
      mas = merge(mas, data);
    }
  } else {
    world.send(0, 0, local_input);
  }

  return true;
}

bool volochaev_s_shell_sort_with_simple_merge_16_mpi::Lab3_16_mpi::post_processing() {
  internal_order_test();

  if (world.rank() == 0) {
    int* ans = reinterpret_cast<int*>(taskData->outputs[0]);
    std::copy(mas.begin(), mas.end(), ans);
  }

  return true;
}
\end{lstlisting}

\end{document}
